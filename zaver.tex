\section*{Závěr}
Změřili jsme účiník a fázový posuv rezistoru, kondenzátoru a cívky, viz tabulka \ref{t:jednicka}.

Zjistili jsme, že rezistor a kondenzátor lze pro naše účely považovat za ideální.
Cívka má účiník různý od jedné, takže ji nelze považovat za ideální. Vypočítali jsme indukčnost a odpor v sériovém a paralelním náhradním zapojení. V sériovém zapojení
\begin{equation*}
R_S=\SI{720(50)}{\ohm}  \qquad \qquad L_S=\SI{5.1(2)}{\henry}
\end{equation*}
a v paralelním
\begin{equation*}
R_P=\SI{4270(50)}{\ohm} \qquad \qquad L_P=\SI{6.1(2)}{\henry} \,.
\end{equation*}

Změřili jsme účiník sériového a paralelního RC obvodu, viz tabulka \ref{t:RC} a grafy \ref{g:RCucinik} a \ref{g:RCfaze}. Pomocí něj jsme určili odpor $R = \SI{990(5)}{\ohm}$. Odpor jsme změřili i digitálním multimetrem na \SI{982(2)}{\ohm}.

Změřili jsme závislost proudu a výkonu na velikosti kapacity zařazené do sériového RLC obvodu, viz přiložená tabulka a grafy \ref{g:RLCu}, \ref{g:RLCf} a \ref{g:RLCP}.