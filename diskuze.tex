\section*{Diskuze}
Měření digitálním osciloskopem bylo v tomto případě přesnější než analogovým, protože na analogovém jsme měřili ve spodní části rozsahu.
U digitálního wattmetru bylo velkou chybou zatíženo měření proudu ze stejného důvodu.
Nicméně výsledky získané oběma přístroji se poměrně přesně shodují.


U rezistoru nám podle očekávání vyšel účiník 1.
U kondenzátoru vyšel téměř přesně 0, takže ho můžeme považovat za ideální a jeho vlastní vodivost zanedbat.
U cívky vyšel účiník \num{0.41}, takže má nezanedbatelný odpor.

Měření odporu pomocí účiníku RC obvodu bylo přesnější v paralelním zapojení, a to zejména kvůli již zmiňované nepřesnosti měření proudu. Vypočtené odpory se v rámci chyby přibližně shodují pro všechna měření a shodují se i s hodnotou naměřenou multimetrem.
Měření multimetrem považujeme v tomto případě za přesnější.