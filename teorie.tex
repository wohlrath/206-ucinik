\section*{Teoretická část}

Střední hodnota výkonu v obvodu protékaným střídavým proudem harmonického průběhu nezáleží jen na efektivních hodnotách proudu a napětí, ale také na jejich vzájemném fázovém posuvu \cite{skripta}
\begin{equation} \label{e:ucinik}
P=\frac{1}{T}\int_{0}^{T}u(t)i(t)dt=\frac{U_0I_0}{2}\cos \varphi = UI \cos \varphi \,,
\end{equation}
kde $U_0$, $I_0$ jsou špičkové hodnoty napětí a proudu, $U$, $I$ efektivní a $\varphi$ je fázový posuv proudu vůči napětí.
Výraz $(\cos \varphi)$ se nazývá účiník.

Na rezistoru nedochází k posuvu proudu a napětí, proto je účiník rovný jedné.

Na kondenzátoru dochází k posuvu $\varphi = -\pi/2$ a tedy účiník bude rovný nule. Vlastnosti reálného kondenzátoru se téměř neliší od ideálního.

Na ideální indukčnosti dochází k posuvu $\varphi = \pi/2$ a účiník by měl být rovný nule. Reálné cívky mají vlastní odpor a proto se používá náhradní zapojení, kdy reálnou cívku nahradíme ideální indukčností a odporem buď seriově nebo paralelně tak, aby výsledná impedance byla stejná.

Pro sériové zapojení indukčnosti $L_S$ a odporu $R_S$ platí \cite{skripta}
\begin{equation} \label{e:ser}
R_S = \frac{U}{I}\frac{1}{\sqrt{1+\tan^2\varphi}} \qquad \qquad L_S = \frac{1}{\omega}\frac{U}{I} \sqrt{\frac{\tan^2\varphi}{1+\tan^2\varphi}}  \,.
\end{equation}

Pro paralelní zapojení indukčnosti $L_P$ a odporu $R_P$ platí \cite{skripta}
\begin{equation} \label{e:par}
R_P=\frac{U}{I} \sqrt{1+\tan^2\varphi} \qquad \qquad L_P=\frac{1}{\omega}\frac{U}{I} \sqrt{\frac{1+\tan^2\varphi}{\tan^2\varphi}} \,.
\end{equation}

Pokud spojíme sériově, resp. paralelně rezistor a kondenzátor o známé kapacitě $C$ a změříme účiník, můžeme určit velikost odporu $R_s$, resp. $R_p$ pomocí vztahů
\begin{align}
\label{e:RCs}
%\begin{split}
 R_s &= \frac{P}{I^2} \,,
\\ \label{e:RCp}
 R_p &= \frac{U^2}{P} \,.
%\end{split}
\end{align}